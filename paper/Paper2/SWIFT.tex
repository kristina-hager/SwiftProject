\documentclass[10pt,twocolumn]{article} 
\usepackage{simpleConference}
\usepackage{times}
\usepackage{graphicx}
\usepackage{amssymb}
\usepackage{url,hyperref}

\begin{document}

\title{Subterranean WiFi}

\author{Amy Reed, Kristina Hager\\
Mobile Computing EE382V\\
University of Texas at Austin\\
\\
}

\maketitle
\thispagestyle{empty}

\begin{abstract}
   This is a simple sample of a document created using \LaTeX
   (specifically pdflatex)
   that includes a figure from the Vergil visual editor for Ptolemy II
   that was created by printing to the Acrobat Distiller to get a PDF file.
   It also illustrates a simple two-column conference paper style,
   and use of bibtex to handle bibliographies.
\end{abstract}
\section{Introduction}
Research takes scientists to places difficult to access to retrieve their study data. 
For this project, we will consider a common problem scientists face when doing research in caves involving sensors and associated data loggers. 
In-cave data loggers (sensors) collect data regarding CO2 levels, cave-drip water, temperature, humidity, and more. 
The most common method of retrieving this data from the data loggers in caves is via a human entering the cave, traveling to the collection site, and manually retrieving it. 
This method is very slow, inconvenient, and in some cases quite perilous. 
Furthermore, data loggers can and do malfunction.
A scientist will not know that her data logger has malfunctioned until she is able to visit the datalogger.

We propose a more automatic means of data retrieval from in-cave loggers. 
We propose to ferry data back to the surface using Wifi Direct enabled devices. 
This technology is now common in many mobile devices.
This would be a far more convenient, effective, and safe mechanism for data retrieval versus current techniques.  
The Wifi Direct standard provides infrastructure mechanisms for discovery, addressing, and security which could provide a simple and power effective means of setting up the network and transferring data. 

We will focus on two key questions in this paper regarding the potential of the Wifi Direct technology for this application.
First, we will investigate the range of the Wifi Direct technology in a cave environment.
WiFi Direct theoretically provides a range of up to 200m which is greater than that of Bluetooth technology. 
However, we would like to determine the potential range of Wifi Direct specifically in a cave environment which has different characteristics from a usual surface environment. 
Next, we will create a prototype application for ferrying data using Wifi Direct as provided by the Android platform. 
However, the still nascent implementation of the WiFi Direct APIs will pose some limitations for our use case.
The current implementation of Wifi Direct on Android devices does not allow a device to be simultaneously connected to two WifiDirect groups.
Since the data will need to travel from device to device, each device will need to participate in one or two groups. 
Since the connection to those groups cannot be simultaneous, our prototype application will manage the disconnect from one group and connect to the next group automatically.
In addition, the current wireless protected setup implementation requires the human user of the device to approve any new connections.
However, once a device has gained user permission once to a join a group, the Android platform "remembers" that permission and allows the device to automatically rejoin that group.
Since we want our devices to be able to operate autonomously underground, we will show how to accommodate the approval issue by pre-configuring the devices in their planned sequence.
 
For the first phase of this project, we created a prototype application for Android devices. 
We used this application to test the ability to make Wifi Direct connections within Whirlpool Cave, a popular cave in south Austin with a variety of passage characteristics. 
The ultimate goal of this initial application and the cave excursion is to determine if the range of Wifi Direct is sufficient to support passing messages out of the cave.
The results of this initial testing showed that WiFi Direct had significant range in areas where direct line of sight was achievable between devices.
We will discuss these experiments in a subsequent section of the paper.

For the second phase of the project, we will develop a second Android application to demonstrate how to pass a sample data file from device to device in a manner similar to how data would be ferried out of a cave.
While WiFi Direct can easily be used to transfer data files within a single WiFi Direct group, our application data transfer from peer to peer that extend beyond the range of what a single group can provide.
Ultimately the second application will determine if passing information from peer to peer to ferry the data from the cave is achievable. 
In addition to the key concern that the application could support multiple WiFi direct groups on the same device, we have concerns about timing of discovery and connectivity.
These must be implemented in such a way that a single wireless interface on the device can be shared.

We have not yet finished our research project. 
This paper serves to cover the progress that has been made thus far into answering the questions outlined in this introduction. 
% The remainder of this paper is structured as follows, section x�.

\section{Cave Research and Data Loggers or whatever we want to call this section}

Test citation \cite{mo12}

\section{WiFi Direct}
In today's world, wireless connectivity is, for the most part taken for granted. 
There are still a few circumstances that remind us that the Internet and network connectivity are not guaranteed. 
Whether due to physical or technological constraints occasionally device connectivity cannot be achieved. 
The WiFi Direct standard is one means of potentially filling that gap. 
In some use cases, broader connectivity to infrastructure is not required and there is value to simple device to device communication. 
This is Wifi Direct's niche, those circumstances where a broader infrastructure is either unavailable or unnecessary, and device to device communication is needed.

Wifi Direct is a standard developed by the WiFi Alliance which provides a mechanism for peer to peer communication in the absence of a dedicated wireless infrastructure. 
This standard has several benefits, first and foremost being that it does not require a dedicated wireless access point http://www.techradar.com/us/news/phone-and-communications/mobile-phones/wi-fi-direct-what-it-is-and-why-you-should-care-1065449-keeping this link reference until we have a bibliography or references section, therefore users don't have worry about a DHCP server or other infrastructure pieces to enable their communication. 
In addition, this standard runs on typical wireless hardware found in all mobile devices, and is supported by most of the major mobile platforms such as Android 4.0, IOS 7, Blackberry, Windows 8, and even Xbox. 
Speed and range for WiFi Direct are those of typical WiFi devices and can operation at ---todo get number---Mbs and up to 200 meters depending on the environment. 
Security for WiFi direct utilizes WiFi Protected Setup and WPA2 to ensure security of the communications over the peer to peer network. 
Since power management is especially important for mobile devices WiFi direct includes power management mechanisms that can reduce power consumption for devices regardless of role.
All of these benefits makes WiFi Direct an ideal consideration to utilize as the technology for ferrying data from an underground cave.

The basic concepts of WiFi Direct are as follows. At the core of Wifi Direct is the WiFi Direct Group. This basically functions as an infrastructure basic service set(BSS). 
All components that can connect into a wireless medium in a network are referred to as stations. 
The basic service set (BSS) is a set of all stations that can communicate with each other. 
An infrastructure BSS includes both access points and stations in a wireless connection scenario.
--en.wikipedia.org.wiki.Wireless.LAN--
All WiFi Direct devices must be capable of becoming the group owner. Within a group, a single device takes on this role. 
The Group Owner is responsible for controlling which devices are allowed to join a group, when the group is started and terminated, BSS functionality, Wi-Fi Protected Setup Internal Registrar functionality, and communication between Clients in the Group. 
The Group owner decides if the group is temporary or persistent, a persistent group may be formed again without reinitiating WPS.
Group owners may optionally provide features such as simultaneous (concurrent) connection with an infrastructure network and sharing of that infrastructure connection. In addition to being able to execute the group owner role, WiFi Direct devices must also be capable of group ownership negotiation, discovery, and power management functions.
 
Devices adhering to the standard enter into a discovery phase where peers within range are found. 
Once a list of peers is retrieved, users can select to connect to one of these devices.
Devices wishing to connect may either form a new group or connect to an existing group. 
As part of the connection process WiFi Protected Setup is used to obtain credentials and authenticate the WiFi Direct device. 

A device obtains the role of group owner through one of three mechanisms [what are these]. 
Once the group owner is established, devices may open socket connections to communicate with other devices within the group.
1.	Discovery
2.	Wifi Protected Setup
3.	Group Ownership

\section{Android APIs For WiFi Direct}
For this project, Android APIs were used to evaluate WiFi Direct for the Swift use case. 
Android provides three main areas within their interface. Methods that allow you to discover, request, and connect to peers are defined in the WifiP2pManager class.  
The WiFi manager class provides methods to allow you to interact with the Wi-Fi hardware on your device to do things like discover and connect to peers. 
Listeners that allow you to be notified of the success or failure of WifiP2pManager method calls. 
When calling WifiP2pManager methods, each method can receive a specific listener passed in as a parameter. 
WifiP2pManager methods let you pass in a listener, so that the Wi-Fi P2P framework can notify your activity of the status of a call. 
Intents that notify you of specific events detected by the Wi-Fi P2P framework, such as a dropped connection or a newly discovered peer. 
The Wi-Fi P2P APIs define intents that are broadcast when certain Wi-Fi P2P events happen, such as when a new peer is discovered or when a device's Wi-Fi state changes. 
You can register to receive these intents in your application by creating a broadcast receiver that handles these intents: ---tobecompleted----

\section{Experimental Setups}
a.	Physical Environment
b.	Datalogger
c.	Range
i.	Software
---This section needs to be rewritten it is copy-paste from reference, but a placeholder----The first application which was build was based on a WiFi Direct example provided by Android ADT. 
Request permission to use the Wi-Fi hardware on the device and also declare your application to have the correct minimum SDK version in the Android manifest. 
As a best practice, one should check to see if Wi-Fi P2P is on and supported within the device. 
In the activity's onCreate method, obtain an instance of WifiP2pManager and register the application with the Wi-Fi P2P framework by calling initialize method. 
This method returns a WifiP2pManager.Channel, which is used to connect the application to the Wi-Fi P2P framework. 
Within the application, a broadcast receiver with the WifiP2pManager and WifiP2pManager.
Channel objects along with a reference to your activity. 
This allows your broadcast receiver to notify your activity of interesting events and update it accordingly. 
It also lets you manipulate the device's Wi-Fi state if necessary. 
The application should create an intent filter and add the intents which broadcast receiver checks for. 
Next the application should initiate device discovery which will asynchronously return a list of peers. 
The application can then choose a peer and connect via a socket to transfer data.
d.	Message Passing
i.	Software

\section{Results}

\section{Related Work}

\section{Conclusions and Future Work}

% \begin{figure}[!b]
%   \begin{center}
%     \includegraphics[width=3.5in]{figure.pdf}
%   \end{center}
% 
%   \caption{\small Figure caption. To get a figure to span two columns, use the environment figure* rather than figure.}
%   \label{fig-label}
% \end{figure}

\section{Using \LaTeX with PDF Figures - Leaving for now for reference}

This is a sample document for use with pdflatex, which is
a program that is included with the Miktex distribution
that directly produces PDF files from \LaTeX sources.
To run \LaTeX on this file, you need the following files:
\begin{enumerate}
\item templatePDF.tex (this file)
\item figure.pdf (the figure file)
\item simpleConference.sty (style file)
\item refs.bib (bibliography file)
\end{enumerate}
\noindent
To create a PDF file, execute the following commands:
\begin{enumerate}
\item pdflatex templatePDF
\item bibtex templatePDF
\item pdflatex templatePDF
\item pdflatex templatePDF
\end{enumerate}
\noindent
Yes (strangely) it is necessary to run pdflatex three times.
The result will be a PDF file (plus several other files that \LaTeX
produces).  You will need a mechanism, of course, for executing
commands on the command line. If you are using Windows, I recommend
installing Cygwin and using its bash shell.

\section{How to Include Vergil Diagrams as Figures}

Suppose you wish to include a figure, like that in figure % \ref{fig-label}.
The simplest mechanism is to install Adobe Acrobat, which includes
a ``printer'' called ``Acrobat Distiller.'' Printing to this printer
creates a PDF file, which can be included in a document as shown
here.  To include Ptolemy II models \cite{cgs13},
just print to the distiller from within Vergil and reference
the PDF file in your \LaTeX document.

There is a bit more work to do, however.
The file that is produced by the distiller represents
a complete page, not the individual figure.
You can open it in using Acrobat (version 5.0 or later),
and select Document $\rightarrow$ Crop Pages from the menu.
In the resulting dialog, check ``Remove White Margins.''
Save the modified PDF file in a file and then reference
it in the \LaTeX file as shown in this example.

An alternative is to generate EPS (encapsulated postscript),
but the process is much more complex and fragile.
I recommend using pdflatex and Adobe Acrobat.

\bibliographystyle{abbrv}
\bibliography{refs}
\end{document}

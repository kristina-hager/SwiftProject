\documentclass[10pt,twocolumn]{article} 
\usepackage{simpleConference}
\usepackage{times}
\usepackage{graphicx}
\usepackage{amssymb}
\usepackage{url,hyperref}

\begin{document}

\title{Subterranean WiFi}

\author{Amy Reed, Kristina Hager\\
\\
Mobile Computing\\
February 25, 2014 \\
\\
University of Texas at Austin\\
Austin, TX, 78660, USA\\
\\
eal@eecs. berkeley.edu\\
}

\maketitle
\thispagestyle{empty}

\begin{abstract}
   This is a simple sample of a document created using \LaTeX
   (specifically pdflatex)
   that includes a figure from the Vergil visual editor for Ptolemy II
   that was created by printing to the Acrobat Distiller to get a PDF file.
   It also illustrates a simple two-column conference paper style,
   and use of bibtex to handle bibligraphies.
\end{abstract}
\section{Introduction}
Research can takes scientists to places difficult to access to retrieve their study data. 
For this project we will consider a common problem faced by scientists researching caves in Central Texas. 
Data regarding CO2 levels, cave-drip water, temperature, humidity, etc is collected by in-cave data loggers (sensors). 
The most common method of retrieving this data from sensors in caves is via a human entering the cave and manually retrieving it. 
This method is very slow, inconvenient, and in some cases quite perilous. 

We want to research application of the Wifi Direct technology in the cave environment for the purpose of ferrying data from in-cave sensors back to the surface for easier access by environmental scientists. 
We propose that this would be a far more convenient, effective and safe mechanism of data retrieval versus current techniques.  The Wifi Direct standard provides infrastructure mechanisms for discovery, addressing, and security which could provide a simple and power effective means of setting up the network and transferring data. 

There are two key questions which will be the focus of this paper which evaluate two potential challenges of the WiFi Direct being used to solve this problem. 
The first question is that of range and connectivity. 
While WiFi Direct theoretically provides for longer ranges than Bluetooth technology, in a subterranean environment these characteristics could change. 
In addition to range, the current, and still nacent implementation of the WiFi Direct APIs may be somewhat limited for use in communication beyond a single WiFi Direct group. 
In addition, from the early research into the APIs, the current wireless protected setup implementation doesn't appear to provide a means to automate connectivity approval.
This issue may be circumvented by the concept of remembered groups, but will require further investigation.

For this project, we will explore and evaluate the Wifi Direct technology and quantify the behavior of Wifi Direct using off the shelf Android devices in actual cave environments. 
The focus of the experiments will be on achievable ranges and data transfer rates if benchmarking is feasible while working within the Android APIs. 
We have created a prototype application on Android devices for the initial phase of our research to use to measure Wifi Direct characteristics with the UT Grotto, a popular cave in Austin with a variety of topologies/profiles to test the connectivity of the technology. 
The ultimate goal of this initial application and our cave excursion is to determine if the range is sufficient to enlist WiFi Direct to support passing messages out of the cave.

Following this experiment, a second Android application will be developed. While we have successfully utilized WiFi direct to transfer data files within a single WiFi Direct group, ultimately we need to determine if this protocol and the available Android APIs are suited to passing information from peer to peer to ferry the data from the cave. 
A key concern is that the distance required to ferry the data the entire lengh of a cave may span multiple WiFi Direct groups. This will require that an application could be developed to support multiple WiFi direct groups on the device. 
In addition, timing of discovery and connectivity must be implemented in such a way that a single wireless interface on the device can be shared.

The current state of our research is incomplete, but this paper serves to cover the progress that has been made thus far into answering the questions outlined in this introduction. 
The remainder of this paper is structured as follows, section x�.

\section{Cave Research and Data Loggers or whatever we want to call this section}
\section{WiFi Direct}
In today's world, wireless connectivity is, for the most part taken for granted. 
There are still a few circumstances though, that remind us that the Internet and network connectivity are not guaranteed. 
There are still a few places due to physical or technological constraints where device connectivity cannot be achieved. 
The WiFi Direct standard is one means of potentially filling that gap. 
In some use cases, broader connectivity to infrastructure is not required and there is value to simple device to device communication. 
This is Wifi Direct's niche, those circumstances where a broader infrastructure is either unavailable or unnecessary, and device to device communication is needed.

Wifi Direct is a standard developed by the WiFi Alliance which provides a mechanism for peer to peer communication in the absence of a dedicated wireless infrastructure. 
This standard has several benefits, first and foremost being that it does not require a dedicated wireless access point http://www.techradar.com/us/news/phone-and-communications/mobile-phones/wi-fi-direct-what-it-is-and-why-you-should-care-1065449-keeping this link reference until we have a bibliography or references section, therefore users don't have worry about a DHCP server or other infrastructure pieces to enable their communication. 
In addition, this standard runs on typical wireless hardware found in all mobile devices, and is supported by most of the major mobile platforms such as Android 4.0, IOS 7, Blackberry, Windows 8, and even Xbox. 
Speed and range for WiFi Direct are those of typical WiFi devices and can operation at ---todo get number---Mbs and up to 200 meters depending on the environment. 
Security for WiFi direct utilizes WiFi Protected Setup and WPA2 to ensure security of the communications over the peer to peer network. 
All of these benefits makes WiFi Direct an ideal consideration to utilize as the technology for ferrying data from an underground cave.

The basic concepts of WiFi Direct are as follows. 
Devices adhering to the standard enter into a discovery phase where peers within range are found. 
A device obtains the role of group owner through one of three mechanisms [what are these]. 
Once the group owner is established, devices may open socket connections to communicate with other devices within the group.
1.	Discovery
2.	Wifi Protected Setup
3.	Group Ownership

\section{Android APIs For WiFi Direct}
For this project, Android APIs were used to evaluate WiFi Direct for the Swift use case. 
Android provides three main areas within their interface. Methods that allow you to discover, request, and connect to peers are defined in the WifiP2pManager class.  
The WiFi manager class provides methods to allow you to interact with the Wi-Fi hardware on your device to do things like discover and connect to peers. 
Listeners that allow you to be notified of the success or failure of WifiP2pManager method calls. 
When calling WifiP2pManager methods, each method can receive a specific listener passed in as a parameter. 
WifiP2pManager methods let you pass in a listener, so that the Wi-Fi P2P framework can notify your activity of the status of a call. 
Intents that notify you of specific events detected by the Wi-Fi P2P framework, such as a dropped connection or a newly discovered peer. 
The Wi-Fi P2P APIs define intents that are broadcast when certain Wi-Fi P2P events happen, such as when a new peer is discovered or when a device's Wi-Fi state changes. 
You can register to receive these intents in your application by creating a broadcast receiver that handles these intents: ---tobecompleted----

\section{Experimental Setups}
a.	Physical Environment
b.	Datalogger
c.	Range
i.	Software
---This section needs to be rewritten it is copy-paste from reference, but a placeholder----The first application which was build was based on a WiFi Direct example provided by Android ADT. 
Request permission to use the Wi-Fi hardware on the device and also declare your application to have the correct minimum SDK version in the Android manifest. 
As a best practice, one should check to see if Wi-Fi P2P is on and supported within the device. 
In the activity's onCreate method, obtain an instance of WifiP2pManager and register the application with the Wi-Fi P2P framework by calling initialize method. 
This method returns a WifiP2pManager.Channel, which is used to connect the application to the Wi-Fi P2P framework. 
Within the application, a broadcast receiver with the WifiP2pManager and WifiP2pManager.
Channel objects along with a reference to your activity. 
This allows your broadcast receiver to notify your activity of interesting events and update it accordingly. 
It also lets you manipulate the device's Wi-Fi state if necessary. 
The application should create an intent filter and add the intents which broadcast receiver checks for. 
Next the application should initiate device discovery which will asynchronously return a list of peers. 
The application can then choose a peer and connect via a socket to transfer data.
d.	Message Passing
i.	Software

\section{Results}

\section{Related Work}

\section{Conclusions and Future Work}

% \begin{figure}[!b]
%   \begin{center}
%     \includegraphics[width=3.5in]{figure.pdf}
%   \end{center}
% 
%   \caption{\small Figure caption. To get a figure to span two columns, use the environment figure* rather than figure.}
%   \label{fig-label}
% \end{figure}

\section{Using \LaTeX with PDF Figures - Leaving for now for reference}

This is a sample document for use with pdflatex, which is
a program that is included with the Miktex distribution
that directly produces PDF files from \LaTeX sources.
To run \LaTeX on this file, you need the following files:
\begin{enumerate}
\item templatePDF.tex (this file)
\item figure.pdf (the figure file)
\item simpleConference.sty (style file)
\item refs.bib (bibiliography file)
\end{enumerate}
\noindent
To create a PDF file, execute the following commands:
\begin{enumerate}
\item pdflatex templatePDF
\item bibtex templatePDF
\item pdflatex templatePDF
\item pdflatex templatePDF
\end{enumerate}
\noindent
Yes (strangely) it is necessary to run pdflatex three times.
The result will be a PDF file (plus several other files that \LaTeX
produces).  You will need a mechanism, of course, for executing
commands on the command line. If you are using Windows, I recommend
installing Cygwin and using its bash shell.

\section{How to Include Vergil Diagrams as Figures}

Suppose you wish to include a figure, like that in figure \ref{fig-label}.
The simplest mechanism is to install Adobe Acrobat, which includes
a ``printer'' called ``Acrobat Distiller.'' Printing to this printer
creates a PDF file, which can be included in a document as shown
here.  To include Ptolemy II models \cite{PtolemyVol1:04},
just print to the distiller from within Vergil and reference
the PDF file in your \LaTeX document.

There is a bit more work to do, however.
The file that is produced by the distiller represents
a complete page, not the individual figure.
You can open it in using Acrobat (version 5.0 or later),
and select Document $\rightarrow$ Crop Pages from the menu.
In the resulting dialog, check ``Remove White Margins.''
Save the modified PDF file in a file and then reference
it in the \LaTeX file as shown in this example.

An alternative is to generate EPS (encapsulated postscript),
but the process is much more complex and fragile.
I recommend using pdflatex and Adobe Acrobat.

\bibliographystyle{abbrv}
\bibliography{refs}
\end{document}
